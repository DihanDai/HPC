\documentclass[11pt]{amsart}

\usepackage{amsmath,amssymb,graphicx,bbm}
\usepackage{amsthm,verbatim}
\usepackage{mathrsfs,mathtools}
\usepackage{enumerate}
\usepackage{listings}
\usepackage[footnotesize,bf]{caption}
\usepackage[left=1.1in,right=1.1in,top=1in]{geometry}

%\usepackage{mathdefs}

%% Patch for amsart date
\usepackage{etoolbox}
\makeatletter
\patchcmd{\@maketitle}
  {\ifx\@empty\@dedicatory}
  {\ifx\@empty\@date \else {\vskip3ex \centering\footnotesize\@date\par\vskip1ex}\fi
   \ifx\@empty\@dedicatory}
  {}{}
\patchcmd{\@adminfootnotes}
  {\ifx\@empty\@date\else \@footnotetext{\@setdate}\fi}
  {}{}{}
\makeatother

\title{HW 3: Loop Analysis Programs}
\author{Dihan Dai}
\date{\today}

\begin{document}
\maketitle
\section*{Question 1}
\noindent\textbf{Reason:} When size of matrix $A$ is large, there are $N$ cache misses for every $i$ loop when calculating $z$, which leads to a total of $N^2$ misses.

\noindent\textbf{Solution: }
\begin{enumerate}[Step.1]
\item Distribute the original loop then interchange the loop for calculating $z$. Use $ij$ loop to calculate $y$, and $ji$ loop to calculate $z$.
\item After the first step, the code haven't achieved target performance. Loop unrolling can be used to further improve the performance. (In my code I unroll the inner loop with $8$ parallel loop bodies.) 
\end{enumerate}
\section*{Question 2}
\noindent\textbf{Reason: } The stride of access is $N$ for this permutation of loops. Accessing array in this way will result in more cache misses. (similar to the reason in question 1).

\noindent\textbf{Solution: }
\begin{enumerate}[Step.1]
  \item Interchange the $ikj$ loop to $ijk$ loop. 
  \item After the first step, the code haven't achieved target performance. Loop unrolling can be used to further improve the performance. (In my code I unroll the inner loop with $4$ parallel loop bodies.) 
\end{enumerate}
\section*{Question 3}
\noindent\textbf{Reason: } The strides of $A$, $B$, $C$ are $0$, $N$ and $1$, respectively. Interchanging the loop doesn't change the stride. Therefore loop tiling (or loop unrolling) needs to be used.

\noindent\textbf{Solution:}
\begin{enumerate}[Step.1]
  \item Introduce submatrices with sizes $32$, and calculate the matrix products for each pair of submatrices from $A$ and $B$. (In the code, I first calculate the $C = BA$ using titling then calculate $C = C^{T}$, which seems to be a little faster than the direct tiling. )
\end{enumerate}
\end{document}
